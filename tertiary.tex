% Options for packages loaded elsewhere
% Options for packages loaded elsewhere
\PassOptionsToPackage{unicode}{hyperref}
\PassOptionsToPackage{hyphens}{url}
\PassOptionsToPackage{dvipsnames,svgnames,x11names}{xcolor}
%
\documentclass[
  12pt,
  letterpaper,
  DIV=11,
  numbers=noendperiod]{scrartcl}
\usepackage{xcolor}
\usepackage{amsmath,amssymb}
\setcounter{secnumdepth}{5}
\usepackage{iftex}
\ifPDFTeX
  \usepackage[T1]{fontenc}
  \usepackage[utf8]{inputenc}
  \usepackage{textcomp} % provide euro and other symbols
\else % if luatex or xetex
  \usepackage{unicode-math} % this also loads fontspec
  \defaultfontfeatures{Scale=MatchLowercase}
  \defaultfontfeatures[\rmfamily]{Ligatures=TeX,Scale=1}
\fi
\usepackage{lmodern}
\ifPDFTeX\else
  % xetex/luatex font selection
  \setmainfont[]{Times New Roman}
\fi
% Use upquote if available, for straight quotes in verbatim environments
\IfFileExists{upquote.sty}{\usepackage{upquote}}{}
\IfFileExists{microtype.sty}{% use microtype if available
  \usepackage[]{microtype}
  \UseMicrotypeSet[protrusion]{basicmath} % disable protrusion for tt fonts
}{}
\makeatletter
\@ifundefined{KOMAClassName}{% if non-KOMA class
  \IfFileExists{parskip.sty}{%
    \usepackage{parskip}
  }{% else
    \setlength{\parindent}{0pt}
    \setlength{\parskip}{6pt plus 2pt minus 1pt}}
}{% if KOMA class
  \KOMAoptions{parskip=half}}
\makeatother
% Make \paragraph and \subparagraph free-standing
\makeatletter
\ifx\paragraph\undefined\else
  \let\oldparagraph\paragraph
  \renewcommand{\paragraph}{
    \@ifstar
      \xxxParagraphStar
      \xxxParagraphNoStar
  }
  \newcommand{\xxxParagraphStar}[1]{\oldparagraph*{#1}\mbox{}}
  \newcommand{\xxxParagraphNoStar}[1]{\oldparagraph{#1}\mbox{}}
\fi
\ifx\subparagraph\undefined\else
  \let\oldsubparagraph\subparagraph
  \renewcommand{\subparagraph}{
    \@ifstar
      \xxxSubParagraphStar
      \xxxSubParagraphNoStar
  }
  \newcommand{\xxxSubParagraphStar}[1]{\oldsubparagraph*{#1}\mbox{}}
  \newcommand{\xxxSubParagraphNoStar}[1]{\oldsubparagraph{#1}\mbox{}}
\fi
\makeatother


\usepackage{longtable,booktabs,array}
\usepackage{calc} % for calculating minipage widths
% Correct order of tables after \paragraph or \subparagraph
\usepackage{etoolbox}
\makeatletter
\patchcmd\longtable{\par}{\if@noskipsec\mbox{}\fi\par}{}{}
\makeatother
% Allow footnotes in longtable head/foot
\IfFileExists{footnotehyper.sty}{\usepackage{footnotehyper}}{\usepackage{footnote}}
\makesavenoteenv{longtable}
\usepackage{graphicx}
\makeatletter
\newsavebox\pandoc@box
\newcommand*\pandocbounded[1]{% scales image to fit in text height/width
  \sbox\pandoc@box{#1}%
  \Gscale@div\@tempa{\textheight}{\dimexpr\ht\pandoc@box+\dp\pandoc@box\relax}%
  \Gscale@div\@tempb{\linewidth}{\wd\pandoc@box}%
  \ifdim\@tempb\p@<\@tempa\p@\let\@tempa\@tempb\fi% select the smaller of both
  \ifdim\@tempa\p@<\p@\scalebox{\@tempa}{\usebox\pandoc@box}%
  \else\usebox{\pandoc@box}%
  \fi%
}
% Set default figure placement to htbp
\def\fps@figure{htbp}
\makeatother


% definitions for citeproc citations
\NewDocumentCommand\citeproctext{}{}
\NewDocumentCommand\citeproc{mm}{%
  \begingroup\def\citeproctext{#2}\cite{#1}\endgroup}
\makeatletter
 % allow citations to break across lines
 \let\@cite@ofmt\@firstofone
 % avoid brackets around text for \cite:
 \def\@biblabel#1{}
 \def\@cite#1#2{{#1\if@tempswa , #2\fi}}
\makeatother
\newlength{\cslhangindent}
\setlength{\cslhangindent}{1.5em}
\newlength{\csllabelwidth}
\setlength{\csllabelwidth}{3em}
\newenvironment{CSLReferences}[2] % #1 hanging-indent, #2 entry-spacing
 {\begin{list}{}{%
  \setlength{\itemindent}{0pt}
  \setlength{\leftmargin}{0pt}
  \setlength{\parsep}{0pt}
  % turn on hanging indent if param 1 is 1
  \ifodd #1
   \setlength{\leftmargin}{\cslhangindent}
   \setlength{\itemindent}{-1\cslhangindent}
  \fi
  % set entry spacing
  \setlength{\itemsep}{#2\baselineskip}}}
 {\end{list}}
\usepackage{calc}
\newcommand{\CSLBlock}[1]{\hfill\break\parbox[t]{\linewidth}{\strut\ignorespaces#1\strut}}
\newcommand{\CSLLeftMargin}[1]{\parbox[t]{\csllabelwidth}{\strut#1\strut}}
\newcommand{\CSLRightInline}[1]{\parbox[t]{\linewidth - \csllabelwidth}{\strut#1\strut}}
\newcommand{\CSLIndent}[1]{\hspace{\cslhangindent}#1}



\setlength{\emergencystretch}{3em} % prevent overfull lines

\providecommand{\tightlist}{%
  \setlength{\itemsep}{0pt}\setlength{\parskip}{0pt}}



 


\usepackage{tcolorbox}
\usepackage{amssymb}
\usepackage{yfonts}
\usepackage{bm}


\newtcolorbox{greybox}{
  colback=white,
  colframe=blue,
  coltext=black,
  boxsep=5pt,
  arc=4pt}
  
\newcommand{\sectionbreak}{\clearpage}

 
\newcommand{\ds}[4]{\sum_{{#1}=1}^{#3}\sum_{{#2}=1}^{#4}}
\newcommand{\us}[3]{\mathop{\sum\sum}_{1\leq{#2}<{#1}\leq{#3}}}

\newcommand{\ol}[1]{\overline{#1}}
\newcommand{\ul}[1]{\underline{#1}}

\newcommand{\amin}[1]{\mathop{\text{argmin}}_{#1}}
\newcommand{\amax}[1]{\mathop{\text{argmax}}_{#1}}

\newcommand{\ci}{\perp\!\!\!\perp}

\newcommand{\mc}[1]{\mathcal{#1}}
\newcommand{\mb}[1]{\mathbb{#1}}
\newcommand{\mf}[1]{\mathfrak{#1}}

\newcommand{\eps}{\epsilon}
\newcommand{\lbd}{\lambda}
\newcommand{\alp}{\alpha}
\newcommand{\df}{=:}
\newcommand{\am}[1]{\mathop{\text{argmin}}_{#1}}
\newcommand{\ls}[2]{\mathop{\sum\sum}_{#1}^{#2}}
\newcommand{\ijs}{\mathop{\sum\sum}_{1\leq i<j\leq n}}
\newcommand{\jis}{\mathop{\sum\sum}_{1\leq j<i\leq n}}
\newcommand{\sij}{\sum_{i=1}^n\sum_{j=1}^n}
	
\KOMAoption{captions}{tableheading}
\makeatletter
\@ifpackageloaded{caption}{}{\usepackage{caption}}
\AtBeginDocument{%
\ifdefined\contentsname
  \renewcommand*\contentsname{Table of contents}
\else
  \newcommand\contentsname{Table of contents}
\fi
\ifdefined\listfigurename
  \renewcommand*\listfigurename{List of Figures}
\else
  \newcommand\listfigurename{List of Figures}
\fi
\ifdefined\listtablename
  \renewcommand*\listtablename{List of Tables}
\else
  \newcommand\listtablename{List of Tables}
\fi
\ifdefined\figurename
  \renewcommand*\figurename{Figure}
\else
  \newcommand\figurename{Figure}
\fi
\ifdefined\tablename
  \renewcommand*\tablename{Table}
\else
  \newcommand\tablename{Table}
\fi
}
\@ifpackageloaded{float}{}{\usepackage{float}}
\floatstyle{ruled}
\@ifundefined{c@chapter}{\newfloat{codelisting}{h}{lop}}{\newfloat{codelisting}{h}{lop}[chapter]}
\floatname{codelisting}{Listing}
\newcommand*\listoflistings{\listof{codelisting}{List of Listings}}
\makeatother
\makeatletter
\makeatother
\makeatletter
\@ifpackageloaded{caption}{}{\usepackage{caption}}
\@ifpackageloaded{subcaption}{}{\usepackage{subcaption}}
\makeatother
\usepackage{bookmark}
\IfFileExists{xurl.sty}{\usepackage{xurl}}{} % add URL line breaks if available
\urlstyle{same}
\hypersetup{
  pdftitle={Tertiary Approach Considered Harmful},
  pdfauthor={Jan de Leeuw},
  colorlinks=true,
  linkcolor={blue},
  filecolor={Maroon},
  citecolor={Blue},
  urlcolor={Blue},
  pdfcreator={LaTeX via pandoc}}


\title{Tertiary Approach Considered Harmful}
\author{Jan de Leeuw}
\date{November 21, 2025}
\begin{document}
\maketitle
\begin{abstract}
TBD
\end{abstract}

\renewcommand*\contentsname{Table of contents}
{
\hypersetup{linkcolor=}
\setcounter{tocdepth}{3}
\tableofcontents
}

\sectionbreak

\textbf{Note:} This is a working manuscript which may be
expanded/updated multiple times. All suggestions for improvement are
welcome. All Rmd, tex, html, pdf, R, and C files are in the public
domain. Attribution will be appreciated, but is not required. The files
can be found at \url{https://github.com/deleeuw/tertiary}

\sectionbreak

\section{Introduction}\label{introduction}

In monotone regression the data is

\begin{itemize}
\tightlist
\item
  the \emph{target}, a numerical vector \(y\) of length \(n\),
\item
  a partial order \(\preceq\) on \(\mathcal{N}:=\{1,2,\cdots,n\}\)
\end{itemize}

We define

\begin{itemize}
\tightlist
\item
  \(i\approx j\) if both \(i\preceq j\) and \(j\preceq i\),
\item
  \(i\prec j\) if \(i\preceq j\) but not \(j\preceq i\).
\end{itemize}

In this paper we will only consider the case of a totalk order, which
means that for all \((i,j)\) either \(i\preceq j\) or \(j\preceq i\) (or
both). If \(i\approx j\) we say the pair \((i,j)\) is a \emph{tie}.
Since being tied is an equivalence relation it partitions
\(\mathcal{N}\) into equivalence classes, called \emph{tie-blocks}.

In the least squares version of monotone regression we minimize the
weighted least squares loss function \begin{equation}
\sigma(x):=\sum_{i=1}^nw_i(x_i-y_i)^2
\end{equation} over all \(x\) for which \(x_i\leq x_j\) if \(i\prec j\).
This definition of the monotone regression problem does not say what to
do with ties. If there are no ties then the \(x_i\) must satisfy a total
order: if \(1\prec\cdots\prec n\) we require \(x_1\leq\cdots\leq x_n\).

If there are ties in the data then the users of a non-metric scaling
programm typically has to choose from various options. The two most
prominent ones, proposed by Kruskal (\citeproc{ref-kruskal_64a}{1964a})
and Kruskal (\citeproc{ref-kruskal_64b}{1964b}), are

\begin{itemize}
\tightlist
\item
  Primary Approach: \(x_i\leq x_j\) if \(i\prec j\),
\item
  Secondary Approach: \(x_i\leq x_j\) if \(i\preceq j\).
\end{itemize}

In the primary approach there are no constraints within tie-blocks, in
the secondary approach it follows that we require \(x_i=x_j\) if
\(i\approx j\). Also see Guttman (\citeproc{ref-guttman_68}{1968}) for
an extensive discussion of these options.

Kruskal (\citeproc{ref-kruskal_64a}{1964a}) showed that primary approach
monotone regression can be solved by redefining the constraints.

\begin{itemize}
\tightlist
\item
  Primary Approach Redux: \(x_i\leq x_j\) if \(i\prec j\) and if
  \((i\approx j)\wedge (y_i<y_j)\).
\end{itemize}

Thus \(y\) is used to order the indices within tie-blocks, and there is
no pair \((i,j)\) with \(i\approx j\) and \(y_i=y_j\) the resulting
constraints on \(x\) define a total order. Kruskal does not give an
explicit rule to deal with \((i\approx j)\wedge (y_i=y_j)\), but seems
to suggest to complete the total order by requiring either
\(x_i\leq x_j\) or \(x_j\leq x_i\). Which one of the two choices we make
does not matter for the outcome of the monotone regression.

We need some additional notation for solving the monotone regression
problem with the secondary approach to ties. Suppose there are \(m<n\)
tie-blocks. Compute the tie-block weighted averages \(\overline y_k\).
Also compute the tie-block weights \(\overline w_k\) as the sum of the
\(w_i\) in the tie-block.Then minimize \[
\sigma(\overline x):=\sum_{k=1}^m\overline w_k(\overline x_k-\overline y_k)^2
\] with the constraints

\begin{itemize}
\tightlist
\item
  Secondary Approach Redux: \(\overline x_k\leq\overline x_\ell\) if
  \(k<\ell\).
\end{itemize}

See De Leeuw (\citeproc{ref-deleeuw_A_77}{1977}) for explicit proofs of
the optimality of the two redux versions.

\begin{itemize}
\tightlist
\item
  Tertiary Approach: \(\overline x_k\leq\overline x_\ell\) if
  \(I_k\prec I_\ell\).
\end{itemize}

Suppose there are only \(m<n\) different values in \(y\). Define the
tie-block averages \(\overline y_k\) as the weighted average of the
\(y_i\) in tie-block \(k\).

the tertiary approach is in De Leeuw
(\citeproc{ref-deleeuw_A_77}{1977}). All three approaches are
implemented as options in the smacof package (De Leeuw and Mair
(\citeproc{ref-deleeuw_mair_A_09c}{2009}), Mair, Groenen, and De Leeuw
(\citeproc{ref-mair_groenen_deleeuw_A_22}{2022})).

\section{The Tertiary Approach}\label{the-tertiary-approach}

If there are only a few ties and there is a good fit the difference
between the three approaches will be small. But in general there are
several problems with the tertiary approach, and users of the smacof
program should think twice before using it.

It is shown in De Leeuw (\citeproc{ref-deleeuw_A_77}{1977}) that the
solution of is given by \[
y_i=\overline y_j+(x_i-\overline x_j)
\]

In the first place the monotone regression solution with the tertiary
approach to ties may produce a vector \(y\) which is

\sectionbreak

\section{References}\label{references}

The first is to ignore ties.

\[
y_i=\overline{y}_j+(x_i-\overline{x}_j)
\]

Example: two classes, equal number of elements. If the two group means
are out of order \(\overline{x}_1>\overline x_2\) we have \[
\overline{y}_1=\overline{y}_2=\tfrac12(\overline{x}_1+\overline{x}_2)
\] Thus for \(i\in I_1\) \[
y_i=\tfrac12(\overline{x}_1+\overline{x}_2)+(x_i-\overline{x}_1)=x_i-\tfrac12(\overline x_1-\overline x_2)
\] Thbus \(y_i<0\) if \(x_i<\tfrac12(\overline x_1-\overline x_2)\)

For \(i\in I_2\) \[
y_i=\tfrac12(\overline{x}_1+\overline{x}_2)+(x_i-\overline{x}_2)=x_i+\tfrac12(\overline x_1-\overline x_2)\geq 0.
\] \(x=(1,9,1,3)\) Then \(\overline x_1=5\) and overline
\(\overline x_2=2\). Thus \(\overline{x}_1-\overline{x}_2=3\) and \(y\)
is \((-.5,7.5,2.5,4.5)\)

\phantomsection\label{refs}
\begin{CSLReferences}{1}{0}
\bibitem[\citeproctext]{ref-deleeuw_A_77}
De Leeuw, J. 1977. {``Correctness of Kruskal's Algorithms for Monotone
Regression with Ties.''} \emph{Psychometrika} 42: 141--44.

\bibitem[\citeproctext]{ref-deleeuw_mair_A_09c}
De Leeuw, J., and P. Mair. 2009. {``{Multidimensional Scaling Using
Majorization: SMACOF in R}.''} \emph{Journal of Statistical Software} 31
(3): 1--30. \url{https://www.jstatsoft.org/article/view/v031i03}.

\bibitem[\citeproctext]{ref-guttman_68}
Guttman, L. 1968. {``{A General Nonmetric Technique for Fitting the
Smallest Coordinate Space for a Configuration of Points}.''}
\emph{Psychometrika} 33: 469--506.

\bibitem[\citeproctext]{ref-kruskal_64a}
Kruskal, J. B. 1964a. {``{Multidimensional Scaling by Optimizing
Goodness of Fit to a Nonmetric Hypothesis}.''} \emph{Psychometrika} 29:
1--27.

\bibitem[\citeproctext]{ref-kruskal_64b}
---------. 1964b. {``{Nonmetric Multidimensional Scaling: a Numerical
Method}.''} \emph{Psychometrika} 29: 115--29.

\bibitem[\citeproctext]{ref-mair_groenen_deleeuw_A_22}
Mair, P., P. J. F. Groenen, and J. De Leeuw. 2022. {``{More on
Multidimensional Scaling in R: smacof Version 2}.''} \emph{Journal of
Statistical Software} 102 (10): 1--47.
\url{https://www.jstatsoft.org/article/view/v102i10}.

\end{CSLReferences}




\end{document}
